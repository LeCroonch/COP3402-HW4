\documentclass{article}
\usepackage{amsmath}
\usepackage{color,pxfonts,fix-cm}
\usepackage{latexsym}
\usepackage[mathletters]{ucs}
\DeclareUnicodeCharacter{46}{\textperiodcentered}
\DeclareUnicodeCharacter{58}{$\colon$}
\DeclareUnicodeCharacter{8226}{$\bullet$}
\DeclareUnicodeCharacter{62}{\textgreater}
\usepackage[T1]{fontenc}
\usepackage[utf8x]{inputenc}
\usepackage{pict2e}
\usepackage{wasysym}
\usepackage[english]{babel}
\usepackage{tikz}
\pagestyle{empty}
\usepackage[margin=0in,paperwidth=612pt,paperheight=792pt]{geometry}
\begin{document}
\definecolor{color_29791}{rgb}{0,0,0}
\begin{tikzpicture}[overlay]\path(0pt,0pt);\end{tikzpicture}
\begin{picture}(-5,0)(2.5,0)
\put(167.403,-116.695){\fontsize{17.2154}{1}\usefont{T1}{cmr}{m}{n}\selectfont\color{color_29791}Details on the PL/0 Code Generator}
\put(197.469,-138.613){\fontsize{17.2154}{1}\usefont{T1}{cmr}{m}{n}\selectfont\color{color_29791}( \$Revision: 1.11 \$ )}
\put(251.65,-168.924){\fontsize{11.9552}{1}\usefont{T1}{cmr}{m}{n}\selectfont\color{color_29791}Gary T . Lea v ens}
\put(248.027,-182.872){\fontsize{11.9552}{1}\usefont{T1}{cmr}{m}{n}\selectfont\color{color_29791}Lea v ens@ucf.edu}
\put(243.532,-207.067){\fontsize{11.9552}{1}\usefont{T1}{cmr}{m}{n}\selectfont\color{color_29791}No v ember 15, 2023}
\put(272.4651,-246.765){\fontsize{9.9626}{1}\usefont{T1}{cmr}{b}{n}\selectfont\color{color_29791}Abstract}
\put(99.21706,-265.118){\fontsize{9.9626}{1}\usefont{T1}{cmr}{m}{n}\selectfont\color{color_29791}This document gi v es some details about code generation for the PL/0 language.}
\put(57.00006,-300.138){\fontsize{14.3462}{1}\usefont{T1}{cmr}{b}{n}\selectfont\color{color_29791}1 Intr oduction}
\put(57.00006,-324.978){\fontsize{10.9091}{1}\usefont{T1}{cmr}{m}{n}\selectfont\color{color_29791}The fourth (and last) part of the project in COP 3402 is to b uild a code generator for part of PL/0. This}
\put(57.00006,-338.527){\fontsize{10.9091}{1}\usefont{T1}{cmr}{m}{n}\selectfont\color{color_29791}document gi v es some details about ho w to do that and some hints.}
\put(57.00006,-373.547){\fontsize{14.3462}{1}\usefont{T1}{cmr}{b}{n}\selectfont\color{color_29791}2 What to Read}
\put(57.00006,-398.387){\fontsize{10.9091}{1}\usefont{T1}{cmr}{m}{n}\selectfont\color{color_29791}A good e xplanation of code generation is found in the book Modern Compiler Implementation in J ava [1],}
\put(57.00009,-411.937){\fontsize{10.9091}{1}\usefont{T1}{cmr}{m}{n}\selectfont\color{color_29791}in which we recommend reading chapters 6–12.}
\put(73.9361,-425.486){\fontsize{10.9091}{1}\usefont{T1}{cmr}{m}{n}\selectfont\color{color_29791}Y ou might also w ant to read Systems Softwar e: Essential Concepts [2] chapter 6.}
\put(57.00009,-460.506){\fontsize{14.3462}{1}\usefont{T1}{cmr}{b}{n}\selectfont\color{color_29791}3 Ov er view}
\put(57.00009,-485.346){\fontsize{10.9091}{1}\usefont{T1}{cmr}{m}{n}\selectfont\color{color_29791}The PL/0 language itself is described in the PL/0 Manual , which is a v ailable in the files section of W eb-}
\put(57.00006,-498.895){\fontsize{10.9091}{1}\usefont{T1}{cmr}{m}{n}\selectfont\color{color_29791}courses. The PL/0 Manual defines the grammar of the language and its semantics.}
\put(73.93606,-512.444){\fontsize{10.9091}{1}\usefont{T1}{cmr}{m}{n}\selectfont\color{color_29791}The follo wing subsections specify the interf ace between the Unix operating system (as on Eustis) and}
\put(57.00006,-525.993){\fontsize{10.9091}{1}\usefont{T1}{cmr}{m}{n}\selectfont\color{color_29791}the compiler as a program.}
\put(57.00006,-555.801){\fontsize{11.9552}{1}\usefont{T1}{cmr}{b}{n}\selectfont\color{color_29791}3.1 Inputs}
\put(57.00006,-576.7141){\fontsize{10.9091}{1}\usefont{T1}{cmr}{m}{n}\selectfont\color{color_29791}The compiler will be passed a single file name on the command line, and the command line also include}
\put(57.00006,-590.2631){\fontsize{10.9091}{1}\usefont{T1}{cmr}{m}{n}\selectfont\color{color_29791}one of tw o options (which are described in Section 3.2).}
\put(73.93607,-603.812){\fontsize{10.9091}{1}\usefont{T1}{cmr}{m}{n}\selectfont\color{color_29791}The file name is the name of a file that contains the input PL/0 program to be compiled. Note that this}
\put(57.00006,-617.361){\fontsize{10.9091}{1}\usefont{T1}{cmr}{m}{n}\selectfont\color{color_29791}input program file is not necessarily le g al according to the semantics of PL/0}
\put(391.8941,-613.403){\fontsize{7.9701}{1}\usefont{T1}{cmr}{m}{n}\selectfont\color{color_29791}1}
\put(396.3771,-617.361){\fontsize{10.9091}{1}\usefont{T1}{cmr}{m}{n}\selectfont\color{color_29791}; for e xample it might do a}
\put(57.00006,-630.911){\fontsize{10.9091}{1}\usefont{T1}{cmr}{m}{n}\selectfont\color{color_29791}di vision by 0. F or e xample, if the file name ar gument is hw4-vmtest1.pl0 (and both the compiler e x e-}
\put(57.00006,-644.46){\fontsize{10.9091}{1}\usefont{T1}{cmr}{m}{n}\selectfont\color{color_29791}cutable, ./compiler , and the file hw4-vmtest1.pl0 are in the current directory), then the follo wing}
\put(57.00006,-658.009){\fontsize{10.9091}{1}\usefont{T1}{cmr}{m}{n}\selectfont\color{color_29791}command line (gi v en to the shell on Eustis)}
\put(102.8191,-677.344){\fontsize{10.9091}{1}\usefont{T1}{cmr}{m}{n}\selectfont\color{color_29791}./compiler hw4-vmtest1.pl0}
\end{picture}
\begin{tikzpicture}[overlay]
\path(0pt,0pt);
\draw[color_29791,line width=0.398pt]
(57pt, -688.39pt) -- (244.197pt, -688.39pt)
;
\end{tikzpicture}
\begin{picture}(-5,0)(2.5,0)
\put(69.653,-695.232){\fontsize{5.9776}{1}\usefont{T1}{cmr}{m}{n}\selectfont\color{color_29791}1}
\put(73.139,-699.041){\fontsize{8.9664}{1}\usefont{T1}{cmr}{m}{n}\selectfont\color{color_29791}The compiler’ s front end and static analysis phases can also handle inputs that do not conform to the language, b ut our tests}
\put(57,-710){\fontsize{8.9664}{1}\usefont{T1}{cmr}{m}{n}\selectfont\color{color_29791}should not ha v e such problems.}
\put(288.273,-739.888){\fontsize{10.9091}{1}\usefont{T1}{cmr}{m}{n}\selectfont\color{color_29791}1}
\end{picture}
\newpage
\begin{tikzpicture}[overlay]\path(0pt,0pt);\end{tikzpicture}
\begin{picture}(-5,0)(2.5,0)
\put(57,-72.95898){\fontsize{10.9091}{1}\usefont{T1}{cmr}{m}{n}\selectfont\color{color_29791}will run the compiler on the program in hw4-vmtest1.pl0 and put the generated machine code into}
\put(57,-86.508){\fontsize{10.9091}{1}\usefont{T1}{cmr}{m}{n}\selectfont\color{color_29791}the file hw4-vmtest1.bof .}
\put(73.937,-100.057){\fontsize{10.9091}{1}\usefont{T1}{cmr}{m}{n}\selectfont\color{color_29791}The same thing can also be accomplished using the make command on Unix:}
\put(102.819,-119.584){\fontsize{10.9091}{1}\usefont{T1}{cmr}{m}{n}\selectfont\color{color_29791}make hw4-vmtest1.bof}
\put(57.00002,-155.464){\fontsize{11.9552}{1}\usefont{T1}{cmr}{b}{n}\selectfont\color{color_29791}3.2 Compiler Options}
\put(57.00002,-176.377){\fontsize{10.9091}{1}\usefont{T1}{cmr}{m}{n}\selectfont\color{color_29791}The compiler’ s main function (in the pro vided file compiler\_main.c ) understands tw o options, both}
\put(57,-189.926){\fontsize{10.9091}{1}\usefont{T1}{cmr}{m}{n}\selectfont\color{color_29791}of which produce normal output on the standard output stream ( stdout ), error output on standard error}
\put(57,-203.475){\fontsize{10.9091}{1}\usefont{T1}{cmr}{m}{n}\selectfont\color{color_29791}output stream ( stderr ), and do not create or af fect the .bof file.}
\put(73.93701,-217.024){\fontsize{10.9091}{1}\usefont{T1}{cmr}{m}{n}\selectfont\color{color_29791}The -l option can be used to produce a list of tok ens in the program file, which can be useful for de-}
\put(57.00002,-230.5731){\fontsize{10.9091}{1}\usefont{T1}{cmr}{m}{n}\selectfont\color{color_29791}b ugging the le x er . The compiler stops after producing this list, without proceeding to parsing.}
\put(73.93701,-244.123){\fontsize{10.9091}{1}\usefont{T1}{cmr}{m}{n}\selectfont\color{color_29791}The -u option can be used to unparse the AST produced by the parser and stop after declaration check-}
\put(57.00002,-257.6721){\fontsize{10.9091}{1}\usefont{T1}{cmr}{m}{n}\selectfont\color{color_29791}ing (without generating code). This can be useful for understanding the AST of a program and for other}
\put(57.00002,-271.2211){\fontsize{10.9091}{1}\usefont{T1}{cmr}{m}{n}\selectfont\color{color_29791}kinds of deb ugging.}
\put(57.00002,-301.1231){\fontsize{11.9552}{1}\usefont{T1}{cmr}{b}{n}\selectfont\color{color_29791}3.3 Running the VM}
\put(57.00002,-322.0361){\fontsize{10.9091}{1}\usefont{T1}{cmr}{m}{n}\selectfont\color{color_29791}The output of the compiler in the binary object file can be used as input to the pro vided VM. The VM is}
\put(57.00002,-335.5851){\fontsize{10.9091}{1}\usefont{T1}{cmr}{m}{n}\selectfont\color{color_29791}located in the vm subdirectory of the pro vided files. This VM essentially the SRM from home w ork 1, with}
\put(57.00002,-349.1341){\fontsize{10.9091}{1}\usefont{T1}{cmr}{m}{n}\selectfont\color{color_29791}a fe w changes:}
\put(75.00002,-371.6501){\fontsize{10.9091}{1}\usefont{T1}{cmr}{m}{n}\selectfont\color{color_29791}•}
\put(78.8182,-371.6501){\fontsize{10.9091}{1}\usefont{T1}{cmr}{m}{n}\selectfont\color{color_29791}}
\put(84.27275,-371.6501){\fontsize{10.9091}{1}\usefont{T1}{cmr}{m}{n}\selectfont\color{color_29791}T racing is no longer the def ault, b ut by using the -t option on the command line one can start trac-}
\put(84.27301,-385.1991){\fontsize{10.9091}{1}\usefont{T1}{cmr}{m}{n}\selectfont\color{color_29791}ing the e x ecution from the v ery be ginning of a program’ s e x ecution. The format of the tracing output}
\put(84.27301,-398.7481){\fontsize{10.9091}{1}\usefont{T1}{cmr}{m}{n}\selectfont\color{color_29791}has also been changed to include more of the runtime stack area.}
\put(75.00001,-421.2641){\fontsize{10.9091}{1}\usefont{T1}{cmr}{m}{n}\selectfont\color{color_29791}•}
\put(78.81819,-421.2641){\fontsize{10.9091}{1}\usefont{T1}{cmr}{m}{n}\selectfont\color{color_29791}}
\put(84.27274,-421.2641){\fontsize{10.9091}{1}\usefont{T1}{cmr}{m}{n}\selectfont\color{color_29791}A PINT instruction w as added to print an inte ger in decimal format, see the SRM manual in the vm}
\put(84.27301,-434.8131){\fontsize{10.9091}{1}\usefont{T1}{cmr}{m}{n}\selectfont\color{color_29791}subdirectory for details.}
\put(73.93701,-457.3291){\fontsize{10.9091}{1}\usefont{T1}{cmr}{m}{n}\selectfont\color{color_29791}Y ou can pass the binary object file that results from compilation, for e xample hw4-vmtest1.bof ,}
\put(57,-470.8781){\fontsize{10.9091}{1}\usefont{T1}{cmr}{m}{n}\selectfont\color{color_29791}to the VM, which is assumed to be named vm/vm , by running the follo wing Unix command, with both the}
\put(56.99998,-484.4271){\fontsize{10.9091}{1}\usefont{T1}{cmr}{m}{n}\selectfont\color{color_29791}VM’ s standard output and error output sent to a file, in this case hw4-vmtest1.myo .}
\put(96.27298,-503.9541){\fontsize{10.9091}{1}\usefont{T1}{cmr}{m}{n}\selectfont\color{color_29791}vm/vm hw4-vmtest1.bof > hw4-vmtest1.myo 2>\&1}
\put(73.93698,-523.4811){\fontsize{10.9091}{1}\usefont{T1}{cmr}{m}{n}\selectfont\color{color_29791}The same thing can also be accomplished using the make command on Unix:}
\put(96.27298,-543.0071){\fontsize{10.9091}{1}\usefont{T1}{cmr}{m}{n}\selectfont\color{color_29791}make hw4-vmtest1.myo}
\put(73.93698,-562.5342){\fontsize{10.9091}{1}\usefont{T1}{cmr}{m}{n}\selectfont\color{color_29791}T o see the VM’ s tracing output, you can either ha v e the VM e x ecute the instruction STRA or you can}
\put(56.99998,-576.0831){\fontsize{10.9091}{1}\usefont{T1}{cmr}{m}{n}\selectfont\color{color_29791}pass the -t option on the command line when running the VM, as in the follo wing.}
\put(96.27299,-595.6101){\fontsize{10.9091}{1}\usefont{T1}{cmr}{m}{n}\selectfont\color{color_29791}vm/vm -t hw4-vmtest1.bof > hw4-vmtest1.myto 2>\&1}
\put(73.93699,-615.1371){\fontsize{10.9091}{1}\usefont{T1}{cmr}{m}{n}\selectfont\color{color_29791}The same thing can also be accomplished using the make command on Unix:}
\put(96.27298,-634.6642){\fontsize{10.9091}{1}\usefont{T1}{cmr}{m}{n}\selectfont\color{color_29791}make hw4-vmtest1.myto}
\put(73.93698,-654.1902){\fontsize{10.9091}{1}\usefont{T1}{cmr}{m}{n}\selectfont\color{color_29791}(That is, the Mak efile uses the suf fix .myto to produce tracing output into the .myto file.)}
\put(73.93698,-667.7402){\fontsize{10.9091}{1}\usefont{T1}{cmr}{m}{n}\selectfont\color{color_29791}(The make command can also mak e the .myo or .myto file without you ha ving to ask it to mak e the}
\put(56.99994,-681.2892){\fontsize{10.9091}{1}\usefont{T1}{cmr}{m}{n}\selectfont\color{color_29791}.bof file first, as mak e will automatically chain these commands together .)}
\put(288.2729,-739.8882){\fontsize{10.9091}{1}\usefont{T1}{cmr}{m}{n}\selectfont\color{color_29791}2}
\end{picture}
\newpage
\begin{tikzpicture}[overlay]\path(0pt,0pt);\end{tikzpicture}
\begin{picture}(-5,0)(2.5,0)
\put(57,-72.95898){\fontsize{11.9552}{1}\usefont{T1}{cmr}{b}{n}\selectfont\color{color_29791}3.4 Outputs}
\put(57,-93.87201){\fontsize{10.9091}{1}\usefont{T1}{cmr}{m}{n}\selectfont\color{color_29791}The normal output of the compiler , when no options are used, goes into the binary object file (with a .bof}
\put(57,-107.421){\fontsize{10.9091}{1}\usefont{T1}{cmr}{m}{n}\selectfont\color{color_29791}suf fix). When the -l or -u options are used (see Section 3.2), the normal output goes to the standard out-}
\put(56.99998,-120.97){\fontsize{10.9091}{1}\usefont{T1}{cmr}{m}{n}\selectfont\color{color_29791}put stream ( stdout ).}
\put(73.93698,-134.519){\fontsize{10.9091}{1}\usefont{T1}{cmr}{m}{n}\selectfont\color{color_29791}Ho we v er , all of the compiler’ s error messages should go to the standard error output stream ( stderr ).}
\put(56.99997,-164.421){\fontsize{11.9552}{1}\usefont{T1}{cmr}{b}{n}\selectfont\color{color_29791}3.5 Exit Code}
\put(56.99997,-185.334){\fontsize{10.9091}{1}\usefont{T1}{cmr}{m}{n}\selectfont\color{color_29791}When the compiler finishes without detecting an y errors, it should e xit with a zero error code; otherwise it}
\put(56.99997,-198.884){\fontsize{10.9091}{1}\usefont{T1}{cmr}{m}{n}\selectfont\color{color_29791}should e xit with a non-zero e xit code.}
\put(56.99997,-228.786){\fontsize{11.9552}{1}\usefont{T1}{cmr}{b}{n}\selectfont\color{color_29791}3.6 A Simple Example}
\put(56.99997,-249.699){\fontsize{10.9091}{1}\usefont{T1}{cmr}{m}{n}\selectfont\color{color_29791}Consider the input in the file hw4-gtest1.pl0 , (note that the suf fix is lo wercase ‘P’, lo wercase ‘L ’, and}
\put(56.99995,-263.248){\fontsize{10.9091}{1}\usefont{T1}{cmr}{m}{n}\selectfont\color{color_29791}the numeral zero, i.e., ‘0’) sho wn in Figure 1, which is included in the hw4-tests.zip file in the files}
\put(56.99994,-276.7971){\fontsize{10.9091}{1}\usefont{T1}{cmr}{m}{n}\selectfont\color{color_29791}section of W ebcourses.}
\put(56.99994,-303.2071){\fontsize{9.9626}{1}\usefont{T1}{cmr}{b}{n}\selectfont\color{color_29791}write 8. \# w r i t e s 8}
\put(195.1589,-332.6971){\fontsize{10.9091}{1}\usefont{T1}{cmr}{m}{n}\selectfont\color{color_29791}Figure 1: The test file hw4-gtest1.pl0 .}
\put(73.93594,-360.454){\fontsize{10.9091}{1}\usefont{T1}{cmr}{m}{n}\selectfont\color{color_29791}Compiling this hw4-gtest1.pl0 , for e xample by using the command make hw4-gtest1.bof ,}
\put(56.99997,-374.0031){\fontsize{10.9091}{1}\usefont{T1}{cmr}{m}{n}\selectfont\color{color_29791}produces a binary object file hw4-gtest1.bof . When run in the VM, for e xample by using the com-}
\put(56.99995,-387.5521){\fontsize{10.9091}{1}\usefont{T1}{cmr}{m}{n}\selectfont\color{color_29791}mand make hw4-gtest1.myo this produces output consisting of the character 8, which matches the}
\put(56.99995,-401.1011){\fontsize{10.9091}{1}\usefont{T1}{cmr}{m}{n}\selectfont\color{color_29791}pro vided file hw4-gtest1.out , as sho wn in Figure 2.}
\put(56.99995,-429.7641){\fontsize{9.9626}{1}\usefont{T1}{cmr}{m}{n}\selectfont\color{color_29791}8}
\put(56.99995,-459.2541){\fontsize{10.9091}{1}\usefont{T1}{cmr}{m}{n}\selectfont\color{color_29791}Figure 2: Expected output (on stdout) from running the compiler on hw4-gtest1.pl0 , and then run-}
\put(56.99997,-472.8031){\fontsize{10.9091}{1}\usefont{T1}{cmr}{m}{n}\selectfont\color{color_29791}ning that binary object file on the VM.}
\put(56.99997,-514.66){\fontsize{11.9552}{1}\usefont{T1}{cmr}{b}{n}\selectfont\color{color_29791}3.7 Pr o vided Dri v er and T ests}
\put(56.99997,-535.5731){\fontsize{10.9091}{1}\usefont{T1}{cmr}{m}{n}\selectfont\color{color_29791}W e pro vide a dri v er (which is in the pro vided file compiler\_main.c ) to run the tests.}
\put(73.93597,-549.1221){\fontsize{10.9091}{1}\usefont{T1}{cmr}{m}{n}\selectfont\color{color_29791}T ests are found in the files named hw4-}
\put(249.331,-551.0311){\fontsize{10.9091}{1}\usefont{T1}{cmr}{m}{n}\selectfont\color{color_29791}*}
\put(255.876,-549.1221){\fontsize{10.9091}{1}\usefont{T1}{cmr}{m}{n}\selectfont\color{color_29791}.pl0 . The e xpected output that results from running the}
\put(56.99998,-562.6721){\fontsize{10.9091}{1}\usefont{T1}{cmr}{m}{n}\selectfont\color{color_29791}binary object file the compiler produces for each test in the VM is found in a file named the same as the}
\put(56.99998,-576.2211){\fontsize{10.9091}{1}\usefont{T1}{cmr}{m}{n}\selectfont\color{color_29791}test input b ut with the suf fix .out . F or e xample, the e xpected output of the PL/0 file hw4-gtest1.pl0}
\put(57,-589.77){\fontsize{10.9091}{1}\usefont{T1}{cmr}{m}{n}\selectfont\color{color_29791}is in the file hw4-gtest1.out .}
\put(57,-619.6721){\fontsize{11.9552}{1}\usefont{T1}{cmr}{b}{n}\selectfont\color{color_29791}3.8 Checking Y our W ork}
\put(57,-640.5851){\fontsize{10.9091}{1}\usefont{T1}{cmr}{m}{n}\selectfont\color{color_29791}Y ou can check your o wn compiler by running the tests using the Unix command on Eustis:}
\put(102.819,-660.1121){\fontsize{10.9091}{1}\usefont{T1}{cmr}{m}{n}\selectfont\color{color_29791}make check-outputs}
\put(73.936,-679.639){\fontsize{10.9091}{1}\usefont{T1}{cmr}{m}{n}\selectfont\color{color_29791}Running the abo v e command will generate files with the suf fix .myo ; for e xample your output from}
\put(57,-693.188){\fontsize{10.9091}{1}\usefont{T1}{cmr}{m}{n}\selectfont\color{color_29791}test hw4-gtest3.pl0 will be put into hw4-gtest3.myo .}
\put(288.273,-739.8881){\fontsize{10.9091}{1}\usefont{T1}{cmr}{m}{n}\selectfont\color{color_29791}3}
\end{picture}
\newpage
\begin{tikzpicture}[overlay]\path(0pt,0pt);\end{tikzpicture}
\begin{picture}(-5,0)(2.5,0)
\put(57,-72.95898){\fontsize{14.3462}{1}\usefont{T1}{cmr}{b}{n}\selectfont\color{color_29791}A Hints}
\put(57,-97.79901){\fontsize{10.9091}{1}\usefont{T1}{cmr}{m}{n}\selectfont\color{color_29791}W e will gi v e more hints in the class’ s lecture and lab sections.}
\put(57,-127.361){\fontsize{11.9552}{1}\usefont{T1}{cmr}{b}{n}\selectfont\color{color_29791}A.1 Deb ugging Code Generation}
\put(57,-148.274){\fontsize{10.9091}{1}\usefont{T1}{cmr}{m}{n}\selectfont\color{color_29791}It is often con v enient to write our o wn PL/0 programs to test specific aspects of the code generator . The}
\put(57,-161.8231){\fontsize{10.9091}{1}\usefont{T1}{cmr}{m}{n}\selectfont\color{color_29791}idea is to try to find what programs cause a problem and to isolate the cause of the problem; for e xample}
\put(57,-175.3721){\fontsize{10.9091}{1}\usefont{T1}{cmr}{m}{n}\selectfont\color{color_29791}by checking the output of write statements in the PL/0 code, you can see what part of the gen\_code}
\put(57,-188.9211){\fontsize{10.9091}{1}\usefont{T1}{cmr}{m}{n}\selectfont\color{color_29791}implementation is lik ely to be the cause of the trouble. In some cases you made need to go into more de-}
\put(57,-202.4711){\fontsize{10.9091}{1}\usefont{T1}{cmr}{m}{n}\selectfont\color{color_29791}tail (as described belo w), b ut writing your o wn (simple) test programs can greatly speed up deb ugging by}
\put(57,-216.0201){\fontsize{10.9091}{1}\usefont{T1}{cmr}{m}{n}\selectfont\color{color_29791}helping you quickly refine your theory of what is going wrong.}
\put(73.936,-229.5691){\fontsize{10.9091}{1}\usefont{T1}{cmr}{m}{n}\selectfont\color{color_29791}Consult the PL/0 Manual (a v ailable in the files section on W ebcourses) for the syntax of PL/0.}
\put(73.93602,-243.1181){\fontsize{10.9091}{1}\usefont{T1}{cmr}{m}{n}\selectfont\color{color_29791}If the problem is in your compiler , then you can use such small test programs to trace your compiler’ s}
\put(57.00002,-256.6671){\fontsize{10.9091}{1}\usefont{T1}{cmr}{m}{n}\selectfont\color{color_29791}e x ecution.}
\put(57.00002,-285.8311){\fontsize{10.9091}{1}\usefont{T1}{cmr}{b}{n}\selectfont\color{color_29791}A.1.1 Deb ugging the Code Generated}
\put(57.00002,-306.7431){\fontsize{10.9091}{1}\usefont{T1}{cmr}{m}{n}\selectfont\color{color_29791}The problem is lik ely to be in the code that your compiler is generating, b ut by looking at the generated}
\put(57.00002,-320.2931){\fontsize{10.9091}{1}\usefont{T1}{cmr}{m}{n}\selectfont\color{color_29791}code using small e xample programs, you can isolate problems to specific functions in your code generator}
\put(57.00002,-333.8421){\fontsize{10.9091}{1}\usefont{T1}{cmr}{m}{n}\selectfont\color{color_29791}(i.e., in gen\_code.c ).}
\put(73.93602,-347.3911){\fontsize{10.9091}{1}\usefont{T1}{cmr}{m}{n}\selectfont\color{color_29791}If you suspect the problem is in the generated code, it is often helpful to see the assembly language}
\put(57.00002,-360.9401){\fontsize{10.9091}{1}\usefont{T1}{cmr}{m}{n}\selectfont\color{color_29791}form of the generated code. T o see the assembly language form of the generated code you can either use}
\put(57.00002,-374.4891){\fontsize{10.9091}{1}\usefont{T1}{cmr}{m}{n}\selectfont\color{color_29791}the -p option of the VM (with a command lik e vm/vm -p mytest.bof ) or you can use the pro vided}
\put(57.00003,-388.0391){\fontsize{10.9091}{1}\usefont{T1}{cmr}{m}{n}\selectfont\color{color_29791}disassembler , with a command lik e:}
\put(96.27303,-406.8751){\fontsize{10.9091}{1}\usefont{T1}{cmr}{m}{n}\selectfont\color{color_29791}vm/disasm mytest.bof > mytest.asm 2>\&1}
\put(73.93603,-425.7111){\fontsize{10.9091}{1}\usefont{T1}{cmr}{m}{n}\selectfont\color{color_29791}The same thing can be accomplished more con v eniently by using the command:}
\put(96.27303,-444.5471){\fontsize{10.9091}{1}\usefont{T1}{cmr}{m}{n}\selectfont\color{color_29791}make mytest.asm}
\put(73.93603,-463.3831){\fontsize{10.9091}{1}\usefont{T1}{cmr}{m}{n}\selectfont\color{color_29791}with the pro vided Mak efile. (This command will automatically generate the mytest.bof file if}
\put(57.00006,-476.9321){\fontsize{10.9091}{1}\usefont{T1}{cmr}{m}{n}\selectfont\color{color_29791}needed.)}
\put(73.93607,-490.4821){\fontsize{10.9091}{1}\usefont{T1}{cmr}{m}{n}\selectfont\color{color_29791}If you find that some code you thought you were generating is missing, check to mak e sure that your}
\put(57.00006,-504.0311){\fontsize{10.9091}{1}\usefont{T1}{cmr}{m}{n}\selectfont\color{color_29791}code generation functions are returning the missing code sequences. Problems can be caused by f ailing}
\put(57.00006,-517.5801){\fontsize{10.9091}{1}\usefont{T1}{cmr}{m}{n}\selectfont\color{color_29791}to assign the results of calling code\_seq\_concat and code\_seq\_add\_to\_end , because code se-}
\put(57.00009,-531.1292){\fontsize{10.9091}{1}\usefont{T1}{cmr}{m}{n}\selectfont\color{color_29791}quences (type code\_seq ) are link ed lists (and C passes ar guments using call-by-v alue).}
\put(57.00008,-560.2932){\fontsize{10.9091}{1}\usefont{T1}{cmr}{b}{n}\selectfont\color{color_29791}A.1.2 T racing the VM’ s Execution of Generated Code}
\put(57.00008,-581.2051){\fontsize{10.9091}{1}\usefont{T1}{cmr}{m}{n}\selectfont\color{color_29791}If you w ould lik e to see the details of ho w the VM is e x ecuting your code, then use the tracing option of}
\put(57.00008,-594.7551){\fontsize{10.9091}{1}\usefont{T1}{cmr}{m}{n}\selectfont\color{color_29791}the VM. Y ou can do this by running your program using the command:}
\put(96.27309,-613.5911){\fontsize{10.9091}{1}\usefont{T1}{cmr}{m}{n}\selectfont\color{color_29791}make mytest.myto}
\put(73.93609,-632.4271){\fontsize{10.9091}{1}\usefont{T1}{cmr}{m}{n}\selectfont\color{color_29791}and then looking at ho w the VM is e x ecuting each instruction.}
\put(57.00009,-661.9891){\fontsize{11.9552}{1}\usefont{T1}{cmr}{b}{n}\selectfont\color{color_29791}A.2 General T ips}
\put(57.00009,-682.9022){\fontsize{10.9091}{1}\usefont{T1}{cmr}{m}{n}\selectfont\color{color_29791}Recursion is your friend (ag ain) and is assumed in the structure of the gen\_code module’ s functions.}
\put(57.00009,-696.4512){\fontsize{10.9091}{1}\usefont{T1}{cmr}{m}{n}\selectfont\color{color_29791}Write code trusting that the functions called w ork properly and concentrate on understanding what each}
\put(57.00009,-710.0001){\fontsize{10.9091}{1}\usefont{T1}{cmr}{m}{n}\selectfont\color{color_29791}function is responsible for doing.}
\put(288.2731,-739.8881){\fontsize{10.9091}{1}\usefont{T1}{cmr}{m}{n}\selectfont\color{color_29791}4}
\end{picture}
\newpage
\begin{tikzpicture}[overlay]\path(0pt,0pt);\end{tikzpicture}
\begin{picture}(-5,0)(2.5,0)
\put(73.936,-72.95898){\fontsize{10.9091}{1}\usefont{T1}{cmr}{m}{n}\selectfont\color{color_29791}An e xample that sho ws ho w to do much of the code generation is pro vided by the FLO A T language,}
\put(57,-86.508){\fontsize{10.9091}{1}\usefont{T1}{cmr}{m}{n}\selectfont\color{color_29791}which is a v ailable from the course’ s e xample code webpage.}
\put(73.936,-100.057){\fontsize{10.9091}{1}\usefont{T1}{cmr}{m}{n}\selectfont\color{color_29791}A good e xample of ho w to do a tree w alk on the ASTs (e.g., to b uild a symbol and check declarations}
\put(57,-113.606){\fontsize{10.9091}{1}\usefont{T1}{cmr}{m}{n}\selectfont\color{color_29791}and identifier uses), is contained in the pro vided unparser and scope\_check modules.}
\put(73.93701,-127.156){\fontsize{10.9091}{1}\usefont{T1}{cmr}{m}{n}\selectfont\color{color_29791}T o find a name in lots of source code, from the Unix command line (or from the MacOS terminal app)}
\put(57.00002,-140.705){\fontsize{10.9091}{1}\usefont{T1}{cmr}{m}{n}\selectfont\color{color_29791}you can use the command grep , as in the follo wing, which searches all of your files ending in .c for the}
\put(57,-154.254){\fontsize{10.9091}{1}\usefont{T1}{cmr}{m}{n}\selectfont\color{color_29791}string kind :}
\put(89.728,-173.781){\fontsize{10.9091}{1}\usefont{T1}{cmr}{m}{n}\selectfont\color{color_29791}grep 'kind'}
\put(168.274,-175.69){\fontsize{10.9091}{1}\usefont{T1}{cmr}{m}{n}\selectfont\color{color_29791}*}
\put(174.819,-173.781){\fontsize{10.9091}{1}\usefont{T1}{cmr}{m}{n}\selectfont\color{color_29791}.c}
\put(57,-193.308){\fontsize{10.9091}{1}\usefont{T1}{cmr}{m}{n}\selectfont\color{color_29791}IDEs and the W indo ws e xplorer pro vide similar commands to search files. Y ou can also use findstr in}
\put(57.00003,-206.857){\fontsize{10.9091}{1}\usefont{T1}{cmr}{m}{n}\selectfont\color{color_29791}W indo ws or Select-String in the W indo ws Po werShell.}
\put(57.00003,-236.759){\fontsize{11.9552}{1}\usefont{T1}{cmr}{b}{n}\selectfont\color{color_29791}A.3 Pr o vided Files}
\put(57.00003,-257.672){\fontsize{10.9091}{1}\usefont{T1}{cmr}{m}{n}\selectfont\color{color_29791}Note that we are pro viding (in the hw4-tests.zip file in the files section on W ebcourses se v eral mod-}
\put(57.00002,-271.221){\fontsize{10.9091}{1}\usefont{T1}{cmr}{m}{n}\selectfont\color{color_29791}ules, including:}
\put(75.00002,-296.725){\fontsize{10.9091}{1}\usefont{T1}{cmr}{m}{n}\selectfont\color{color_29791}•}
\put(78.8182,-296.725){\fontsize{10.9091}{1}\usefont{T1}{cmr}{m}{n}\selectfont\color{color_29791}}
\put(84.27275,-296.725){\fontsize{10.9091}{1}\usefont{T1}{cmr}{m}{n}\selectfont\color{color_29791}The code module (see the pro vided files code.h and code.c ), which has functions (such as}
\put(84.27304,-310.275){\fontsize{10.9091}{1}\usefont{T1}{cmr}{m}{n}\selectfont\color{color_29791}code\_srl ) that can con v eniently create each type of indi vidual machine instruction (these will be}
\put(84.27305,-323.824){\fontsize{10.9091}{1}\usefont{T1}{cmr}{m}{n}\selectfont\color{color_29791}returned as pointers to the type code ) and functions that w ork on sequences of such code elements}
\put(84.27304,-337.373){\fontsize{10.9091}{1}\usefont{T1}{cmr}{m}{n}\selectfont\color{color_29791}(with the type code\_seq ).}
\put(75.00005,-359.889){\fontsize{10.9091}{1}\usefont{T1}{cmr}{m}{n}\selectfont\color{color_29791}•}
\put(78.81823,-359.889){\fontsize{10.9091}{1}\usefont{T1}{cmr}{m}{n}\selectfont\color{color_29791}}
\put(84.27278,-359.889){\fontsize{10.9091}{1}\usefont{T1}{cmr}{m}{n}\selectfont\color{color_29791}The id\_use module pro vides the type id\_use , and (pointers to) id\_use structures are put into}
\put(84.27307,-373.438){\fontsize{10.9091}{1}\usefont{T1}{cmr}{m}{n}\selectfont\color{color_29791}the ASTs by declaration checking, so that the y are a v ailable when generating code for an identifier}
\put(84.27307,-386.987){\fontsize{10.9091}{1}\usefont{T1}{cmr}{m}{n}\selectfont\color{color_29791}usage. Note that an id\_use struct pro vides access to the name’ s id\_attrs using the pro vided}
\put(84.27307,-400.536){\fontsize{10.9091}{1}\usefont{T1}{cmr}{m}{n}\selectfont\color{color_29791}function id\_use\_get\_attrs .}
\put(75.00006,-423.052){\fontsize{10.9091}{1}\usefont{T1}{cmr}{m}{n}\selectfont\color{color_29791}•}
\put(78.81824,-423.052){\fontsize{10.9091}{1}\usefont{T1}{cmr}{m}{n}\selectfont\color{color_29791}}
\put(84.2728,-423.052){\fontsize{10.9091}{1}\usefont{T1}{cmr}{m}{n}\selectfont\color{color_29791}The id\_attrs module pro vides the type id\_attrs and functions that w ork with those attrib utes.}
\put(75.00005,-445.567){\fontsize{10.9091}{1}\usefont{T1}{cmr}{m}{n}\selectfont\color{color_29791}•}
\put(78.81823,-445.567){\fontsize{10.9091}{1}\usefont{T1}{cmr}{m}{n}\selectfont\color{color_29791}}
\put(84.27278,-445.567){\fontsize{10.9091}{1}\usefont{T1}{cmr}{m}{n}\selectfont\color{color_29791}The ast module defines the structure and information in the ASTs. The ASTs are essentially the}
\put(84.27306,-459.116){\fontsize{10.9091}{1}\usefont{T1}{cmr}{m}{n}\selectfont\color{color_29791}same as in home w ork 3, b ut no w ha v e id\_use pointers in all constant and v ariable identifier uses.}
\put(57.00006,-489.019){\fontsize{11.9552}{1}\usefont{T1}{cmr}{b}{n}\selectfont\color{color_29791}A.4 Gradual De v elopment}
\put(57.00006,-509.931){\fontsize{10.9091}{1}\usefont{T1}{cmr}{m}{n}\selectfont\color{color_29791}When writing the code generator , it is useful to b uild the capabilities of the code generator gradually , so}
\put(57.00006,-523.481){\fontsize{10.9091}{1}\usefont{T1}{cmr}{m}{n}\selectfont\color{color_29791}you can test as you mak e progress. T o do that, start with simple e xamples such as nonterminals that ha v e}
\put(57.00006,-537.03){\fontsize{10.9091}{1}\usefont{T1}{cmr}{m}{n}\selectfont\color{color_29791}no productions that generate other nonterminals (lik e identifier e xpressions), and then use these to b uild up}
\put(57.00006,-550.579){\fontsize{10.9091}{1}\usefont{T1}{cmr}{m}{n}\selectfont\color{color_29791}to more comple x e xamples (by combining the generated code sequences for the simpler e xamples). In this}
\put(57.00006,-564.128){\fontsize{10.9091}{1}\usefont{T1}{cmr}{m}{n}\selectfont\color{color_29791}w ay you can deb ug each part of the code generator as you proceed and use the recursi v e tree w alk o v er the}
\put(57.00006,-577.677){\fontsize{10.9091}{1}\usefont{T1}{cmr}{m}{n}\selectfont\color{color_29791}AST to combine the generated code sequences for simple e xamples into more comple x code sequences for}
\put(57.00006,-591.227){\fontsize{10.9091}{1}\usefont{T1}{cmr}{m}{n}\selectfont\color{color_29791}more comple x e xamples.}
\put(73.93706,-604.776){\fontsize{10.9091}{1}\usefont{T1}{cmr}{m}{n}\selectfont\color{color_29791}The follo wing might be a useful order to gradually b uild up to more comple x e xamples (this is to some}
\put(57.00006,-618.325){\fontsize{10.9091}{1}\usefont{T1}{cmr}{m}{n}\selectfont\color{color_29791}e xtent follo wed by the pro vided tests named hw4-gtest}
\put(311.3441,-620.234){\fontsize{10.9091}{1}\usefont{T1}{cmr}{m}{n}\selectfont\color{color_29791}*}
\put(317.8891,-618.325){\fontsize{10.9091}{1}\usefont{T1}{cmr}{m}{n}\selectfont\color{color_29791}.pl0 ).}
\put(75.00006,-640.841){\fontsize{10.9091}{1}\usefont{T1}{cmr}{m}{n}\selectfont\color{color_29791}•}
\put(78.81824,-640.841){\fontsize{10.9091}{1}\usefont{T1}{cmr}{m}{n}\selectfont\color{color_29791}}
\put(84.2728,-640.841){\fontsize{10.9091}{1}\usefont{T1}{cmr}{m}{n}\selectfont\color{color_29791}The skip statement (and basic program bookk eeping for creating a binary object file).}
\put(75.00006,-663.356){\fontsize{10.9091}{1}\usefont{T1}{cmr}{m}{n}\selectfont\color{color_29791}•}
\put(78.81824,-663.356){\fontsize{10.9091}{1}\usefont{T1}{cmr}{m}{n}\selectfont\color{color_29791}}
\put(84.2728,-663.356){\fontsize{10.9091}{1}\usefont{T1}{cmr}{m}{n}\selectfont\color{color_29791}The write statement and numeric literals (hint: implement and use the literal\_table for nu-}
\put(84.27304,-676.905){\fontsize{10.9091}{1}\usefont{T1}{cmr}{m}{n}\selectfont\color{color_29791}meric literals).}
\put(75.00004,-699.421){\fontsize{10.9091}{1}\usefont{T1}{cmr}{m}{n}\selectfont\color{color_29791}•}
\put(78.81822,-699.421){\fontsize{10.9091}{1}\usefont{T1}{cmr}{m}{n}\selectfont\color{color_29791}}
\put(84.27277,-699.421){\fontsize{10.9091}{1}\usefont{T1}{cmr}{m}{n}\selectfont\color{color_29791}The begin statement, so that a program can do more than one thing (such as tw o write statements).}
\put(288.273,-739.888){\fontsize{10.9091}{1}\usefont{T1}{cmr}{m}{n}\selectfont\color{color_29791}5}
\end{picture}
\newpage
\begin{tikzpicture}[overlay]\path(0pt,0pt);\end{tikzpicture}
\begin{picture}(-5,0)(2.5,0)
\put(75,-72.95898){\fontsize{10.9091}{1}\usefont{T1}{cmr}{m}{n}\selectfont\color{color_29791}•}
\put(78.81818,-72.95898){\fontsize{10.9091}{1}\usefont{T1}{cmr}{m}{n}\selectfont\color{color_29791}}
\put(84.27274,-72.95898){\fontsize{10.9091}{1}\usefont{T1}{cmr}{m}{n}\selectfont\color{color_29791}Constant declarations and identifier uses.}
\put(75,-95.474){\fontsize{10.9091}{1}\usefont{T1}{cmr}{m}{n}\selectfont\color{color_29791}•}
\put(78.81818,-95.474){\fontsize{10.9091}{1}\usefont{T1}{cmr}{m}{n}\selectfont\color{color_29791}}
\put(84.27274,-95.474){\fontsize{10.9091}{1}\usefont{T1}{cmr}{m}{n}\selectfont\color{color_29791}V ariable declarations.}
\put(75,-117.99){\fontsize{10.9091}{1}\usefont{T1}{cmr}{m}{n}\selectfont\color{color_29791}•}
\put(78.81818,-117.99){\fontsize{10.9091}{1}\usefont{T1}{cmr}{m}{n}\selectfont\color{color_29791}}
\put(84.27274,-117.99){\fontsize{10.9091}{1}\usefont{T1}{cmr}{m}{n}\selectfont\color{color_29791}Assignment statements (which can use the kinds of e xpressions implemented already , numeric liter -}
\put(84.273,-131.539){\fontsize{10.9091}{1}\usefont{T1}{cmr}{m}{n}\selectfont\color{color_29791}als and identifiers).}
\put(75,-154.055){\fontsize{10.9091}{1}\usefont{T1}{cmr}{m}{n}\selectfont\color{color_29791}•}
\put(78.81818,-154.055){\fontsize{10.9091}{1}\usefont{T1}{cmr}{m}{n}\selectfont\color{color_29791}}
\put(84.27274,-154.055){\fontsize{10.9091}{1}\usefont{T1}{cmr}{m}{n}\selectfont\color{color_29791}Binary e xpressions (such as x + 1 ), which can be used in both assignment and write statements.}
\put(75,-176.57){\fontsize{10.9091}{1}\usefont{T1}{cmr}{m}{n}\selectfont\color{color_29791}•}
\put(78.81818,-176.57){\fontsize{10.9091}{1}\usefont{T1}{cmr}{m}{n}\selectfont\color{color_29791}}
\put(84.27274,-176.57){\fontsize{10.9091}{1}\usefont{T1}{cmr}{m}{n}\selectfont\color{color_29791}Conditions and if-statements.}
\put(75,-199.086){\fontsize{10.9091}{1}\usefont{T1}{cmr}{m}{n}\selectfont\color{color_29791}•}
\put(78.81818,-199.086){\fontsize{10.9091}{1}\usefont{T1}{cmr}{m}{n}\selectfont\color{color_29791}}
\put(84.27274,-199.086){\fontsize{10.9091}{1}\usefont{T1}{cmr}{m}{n}\selectfont\color{color_29791}While loops.}
\put(73.936,-221.601){\fontsize{10.9091}{1}\usefont{T1}{cmr}{m}{n}\selectfont\color{color_29791}As you gradually de v elop the code generator , it is often useful to use “stubs” in your coding, so that}
\put(57,-235.151){\fontsize{10.9091}{1}\usefont{T1}{cmr}{m}{n}\selectfont\color{color_29791}if an e xample turns out to call code generation for some nonterminal that is not yet implemented, you}
\put(57,-248.7){\fontsize{10.9091}{1}\usefont{T1}{cmr}{m}{n}\selectfont\color{color_29791}will kno w about that. A stub can be created for a function in C by ha ving the body of the function call}
\put(57,-262.249){\fontsize{10.9091}{1}\usefont{T1}{cmr}{m}{n}\selectfont\color{color_29791}bail\_with\_error with an appropriate error message. F or e xample, you might use the follo wing stubs}
\put(57,-275.798){\fontsize{10.9091}{1}\usefont{T1}{cmr}{m}{n}\selectfont\color{color_29791}in gen\_code.c :}
\put(59.342,-295.325){\fontsize{10.9091}{1}\usefont{T1}{cmr}{m}{it}\selectfont\color{color_29791}/ / ( S t u b f o r : ) G e n e r a t e c o d e f o r t h e p r o c e d u r e d e c l a r a t i o n s}
\put(57,-308.8741){\fontsize{10.9091}{1}\usefont{T1}{cmr}{m}{n}\selectfont\color{color_29791}code\_seq gen\_code\_proc\_decls(proc\_decls\_t pds)}
\put(57,-322.4231){\fontsize{10.9091}{1}\usefont{T1}{cmr}{m}{n}\selectfont\color{color_29791}\{}
\put(83.182,-335.9731){\fontsize{10.9091}{1}\usefont{T1}{cmr}{m}{n}\selectfont\color{color_29791}bail\_with\_error("TODO: no implementation of gen\_code\_proc\_decls yet!");}
\put(83.182,-349.5221){\fontsize{10.9091}{1}\usefont{T1}{cmr}{b}{n}\selectfont\color{color_29791}return code\_seq\_empty();}
\put(57.00001,-363.0711){\fontsize{10.9091}{1}\usefont{T1}{cmr}{m}{n}\selectfont\color{color_29791}\}}
\put(59.34201,-390.1691){\fontsize{10.9091}{1}\usefont{T1}{cmr}{m}{it}\selectfont\color{color_29791}/ / ( S t u b f o r : ) G e n e r a t e c o d e f o r a p r o c e d u r e d e c l a r a t i o n}
\put(57.00001,-403.7191){\fontsize{10.9091}{1}\usefont{T1}{cmr}{m}{n}\selectfont\color{color_29791}code\_seq gen\_code\_proc\_decl(proc\_decl\_t pd)}
\put(57.00001,-417.2681){\fontsize{10.9091}{1}\usefont{T1}{cmr}{m}{n}\selectfont\color{color_29791}\{}
\put(83.18201,-430.8171){\fontsize{10.9091}{1}\usefont{T1}{cmr}{m}{n}\selectfont\color{color_29791}bail\_with\_error("TODO: no implementation of gen\_code\_proc\_decl yet!");}
\put(83.18201,-444.3661){\fontsize{10.9091}{1}\usefont{T1}{cmr}{b}{n}\selectfont\color{color_29791}return code\_seq\_empty();}
\put(57.00001,-457.9151){\fontsize{10.9091}{1}\usefont{T1}{cmr}{m}{n}\selectfont\color{color_29791}\}}
\put(57.00001,-499.0071){\fontsize{14.3462}{1}\usefont{T1}{cmr}{b}{n}\selectfont\color{color_29791}Refer ences}
\put(57.00001,-523.8481){\fontsize{10.9091}{1}\usefont{T1}{cmr}{m}{n}\selectfont\color{color_29791}[1]}
\put(69.72002,-523.8481){\fontsize{10.9091}{1}\usefont{T1}{cmr}{m}{n}\selectfont\color{color_29791}}
\put(75.17457,-523.8481){\fontsize{10.9091}{1}\usefont{T1}{cmr}{m}{n}\selectfont\color{color_29791}Andre w Appel and Jens P alsber g. Modern Compiler Implementation in J ava: Second Edition . Cam-}
\put(75.17401,-537.3971){\fontsize{10.9091}{1}\usefont{T1}{cmr}{m}{n}\selectfont\color{color_29791}bridge, 2002.}
\put(57.00002,-559.9121){\fontsize{10.9091}{1}\usefont{T1}{cmr}{m}{n}\selectfont\color{color_29791}[2]}
\put(69.72002,-559.9121){\fontsize{10.9091}{1}\usefont{T1}{cmr}{m}{n}\selectfont\color{color_29791}}
\put(75.17458,-559.9121){\fontsize{10.9091}{1}\usefont{T1}{cmr}{m}{n}\selectfont\color{color_29791}Euripides Montagne. Systems Softwar e: Essential Concepts . Cognella Academic Publishing, 2021.}
\put(288.273,-739.8881){\fontsize{10.9091}{1}\usefont{T1}{cmr}{m}{n}\selectfont\color{color_29791}6}
\end{picture}
\end{document}